\documentclass[a4paper]{article}   % list options between brackets
\usepackage{CJK}
\usepackage{fullpage}
\usepackage{graphicx}
\usepackage{graphics}
\usepackage{tabularx}
\usepackage{listings}
\usepackage{rotating}
\usepackage{setspace}
\usepackage{amsmath}
\usepackage{caption}
\usepackage{xcolor}
\usepackage{cite}

\lstset{numbers=left, numberstyle=\small, frame=shadowbox,
xleftmargin=2em,xrightmargin=2em, aboveskip=1em, breaklines,
extendedchars=false, tabsize=4, basicstyle=\linespread{1.0}\ttfamily,
commentstyle=\ttfamily\normal\small, keywordstyle=\color{blue!70}
}

% type user-defined commands here


\begin{document}
\begin{CJK}{UTF8}{gbsn}
\begin{spacing}{1.2}

\title{High Performance Computing\\ Homework 3\\ Report}
\author{Peiyun Hu, 2010011297\\ Department of Computer Science \& Technology}
\maketitle

\section{Environment}
\subsection{CPU}
Intel(R) Core(TM) i7-2630QM CPU @ 2.00GHz
cpu cores	: 4
processors	: 8
\subsection{Compiler}
gcc version 4.6.1 (Ubuntu/Linaro 4.6.1-9ubuntu3)
\subsection{Makefile}
gcc -fopenmp -w -o \$exec\_path -O2 \$src\_path -lm -lrt 
\subsection{Number of Processes}
All the results of parallel programming below is based on 8 processes. 

\section{The Effect of Parallelism on Different Level of Nested Loop}
We will research on the multiplication of matrix $\text m \times \text k$ and matrix $\text k\times \text n$, of which the product is matrix $\text m\times \text n$. 

\subsection{The Place of OpenMP directive}

\subsubsection{Implement}
The place of OpenMP directive is decisive on the performance. For example, if we look into this problem by altering the directive '\# pragma omp for', which did not indicate the schedule strategy differently yet, to different places above each nested for-loop. What is more, this order of nested for-loop is i-loop, j-loop, k-loop successively. Specifically, part of the code is as follow: 

\begin{lstlisting}[language=C, label=lst:i-loop, caption=i-loop Paralleled]
#pragma omp parallel private(i, j, k)
{
#pragma omp for schedule(static)
    for(i=0; i<a_rows; i++)
		for(j=0; j<b_cols; j++)
            for(k=0; k<a_cols; k++)
                res[i*a_cols+j] += v1[i*a_cols+k] * v2[k*b_cols+j];
}
\end{lstlisting}

\begin{lstlisting}[language=C, label=lst:j-loop, caption=j-loop Paralleled]
#pragma omp parallel private(i, j, k)
{
    for(i=0; i<a_rows; i++)
    #pragma omp for schedule(static)
		for(j=0; j<b_cols; j++)
            for(k=0; k<a_cols; k++)
                res[i*a_cols+j] += v1[i*a_cols+k] * v2[k*b_cols+j];
}
\end{lstlisting}

\begin{lstlisting}[language=C, label=lst:k-loop, caption=k-loop Paralleled]
#pragma omp parallel private(i, j, k)
{
    for(i=0; i<a_rows; i++)
		for(j=0; j<b_cols; j++)
		#pragma omp for schedule(static)
            for(k=0; k<a_cols; k++){
            	double tmp = v1[i*a_cols+k] * v2[k*b_cols+j];
            	#pragma omp critical
            	{
                	res[i*a_cols+j] += tmp;
            	}
           	}
}
\end{lstlisting}

Moreover, in the Listing \ref{lst:k-loop}, the res[i*a\_ cols+j] could be visited at the same time by different processes. So, before assigning new value to res[i*a\_ cols+j], we should ensure that only one process is executing this statement, which is implemented using statement \ref{code:critical}.
\begin{align}
\label{code:critical}
\textbf{\# pragma omp critical}
\end{align}

\subsubsection{Result}
The result shown in Table \ref{table:prob1_res} is presented under the condition that the order of loops are i, j, k successively, and what is the strategy of schedule is static. The OpenMP directive is placed before the for-loop, as is shown in Listing \ref{lst:i-loop}, \ref{lst:j-loop} and \ref{lst:k-loop}.

\begin{table}[htbp]
\centering
\begin{tabular}{|c|c|c|c|c|}
\hline The Place of Directive & Serial & i-loop  & j-loop & k-loop \\ 
\hline Time(s) & 8.710679 & 2.510435 & 2.710924 & 322.320779 \\ 
\hline 
\end{tabular}
\caption{Performance with different placement of OpenMP directives in nested loops} \label{table:prob1_res}
\end{table}


As is shown in the Table \ref{table:prob1_res}, placing the directive before i-loop and j-loop will almost achieve the same performance, and nearly $\frac{1}{4}$ compared to serial performance. But owing to the competition and wait among threads, the version of k-loop paralleled has a unimaginable poor performance. 

\textbf{In Conclusion, the OpenMP directive should be placed in the first two loop, for example, i-loop or j-loop.}

\subsection{Loop Order}
\subsubsection{Implement}
We ensure that the first loop is paralleled, and change the loop order. For example, we exchange the order of i-loop with j-loop, or i-loop with k-loop. But, if k is the first loop that is paralleled, the execution should be ensured safe, which means that '\#pragma omp critical' should be placed before specific assignment. In detail, part of the implement is presented in Table \ref{table:prob1_res2}:

\subsubsection{Result}
The result presented in Table \ref{table:prob1_res2} is under the condition of different order of for-loops.
\begin{table}[htbp]
\centering
\begin{tabular}{|c|c|c|c|c|}
\hline The Order of For-loop & i, j, k & j, k, i  & k, j, i \\
\hline Time(Paralleled)(s) & 2.510435 & 3.407517 & 265.025561 \\ 
\hline Time(Without Paralleled)(s) & 9.077633 & 8.509484 & 20.044858 \\
\hline 
\end{tabular}
\caption{Performance with Different Order of Nested Loops} \label{table:prob1_res2}
\end{table}

\textbf{In conclusion, the order of for-loop should be i, j, k or j, i, k, instead of making k-loop the first nested loop. }

\section{The Effect of Parallelism For Different Parameters}
\subsection{Experiment}
In this experiment, we iterate 1000 iterations with 4 threads. When iterating the \textbf{i th loop}, the delay within the dummy function is \textbf{i*100}. 

The parameters of OpenMP directive is presented as in Listing \ref{lst:openmp_paras}:
\begin{lstlisting}[language=C, label=lst:openmp_paras, caption=The parameter for Different Schedule]
#pragma omp parallel for schedule(static, 200)
#pragma omp parallel for schedule(dynamic)
#pragma omp parallel for schedule(guided, 100)
\end{lstlisting}

\subsection{Result}
The results is shown below in Table \ref{table:openmp_paras}.

\begin{table}[htbp]
\centering
\begin{tabular}{|c|c|c|c|}
\hline Schedule & Static & Dynamic & Guided \\ 
\hline Time(s) & 20.016861 & 12.549876 & 16.057561 \\ 
\hline 
\end{tabular} 
\caption{Performance with Different Order of Nested Loops} \label{table:openmp_paras}
\end{table}

As is shown in Table \ref{table:openmp_paras}, we could see that in time consumed using static schedule is more much more than dynamic and guided, while dynamic is less than guided. In this situation, the time  invocating \textbf{dummy} once is increasing as iterating. Consequently, \textbf{if tasks are distributed using static schedule, then the tasks will cost more time when task id increase}. While, \textbf{guided schedule will ensure that the scale of task would decrease as iterating}. Moreover, \textbf{dynamic schedule could distributed more flexibly}, and consume the least time. 

\section{Pizza World Scenario}
\subsection{Experiment}
\subsubsection{Sample Input}
22002 22111 20002 01111 00022 12220 21122 02211 00202 21012 00002 02202 10022 00221 00201 11002 01212 02200 12122 10212 10010 20201 01200 21220 

\subsubsection{Sample Output}
\begin{lstlisting}
Execution time:				 			0.001440
Total number of pizza served: 			56
Total profit of for the day: 		 	-552
Number of customers came and leave: 	13
Rate of customer happiness: 		 	0.993804
18:01 (IIII) (T) (MMOOO OOOOO OOOOO OOOOO) (0,0,0);
18:02 (SSII) (T) (HHMMO OOOOO OOOOO OOOOO) (0,0,0);
18:03 (TTSS) (I) (HHHHO OOOOO OOOOO OOOOO) (0,0,0);
18:04 (IITT) (I) (HHHHO OOOOO OOOOO OOOOO) (2,0,0);
18:05 (IIII) (T) (HHHHM MOOOO OOOOO OOOOO) (4,0,0);
18:06 (SSII) (T) (HHHHH HMMOO OOOOO OOOOO) (4,0,0);
18:07 (TTSS) (T) (HHHHH HHHMM OOOOO OOOOO) (4,0,0);
18:08 (SSTT) (T) (HHHHH HHHHH MOOOO OOOOO) (6,0,0);
18:09 (TTSI) (T) (HHHHH HHHHH HMOOO OOOOO) (8,0,0);
18:10 (SITI) (T) (HHHHH HHHHH HHMOO OOOOO) (10,0,0);
18:11 (TSII) (T) (HHHHH HHHHH HHHMM OOOOO) (11,0,0);
18:12 (STSI) (I) (HHHHH HHHHH HHHHH OOOOO) (12,0,0);
18:13 (TITI) (I) (HHHHH HHHHH HHHHH OOOOO) (13,0,0);
18:14 (IIII) (I) (HHHHH HHHHH HHHHH OOOOO) (15,0,0);
18:15 (IIII) (T) (HHHHH HHHHH HHHHH MMOOO) (13,0,0);
18:16 (SSII) (S) (EEEEH HHHHH HHHHH HHOOO) (11,0,0);
18:17 (TTII) (T) (EEEEH HHHHH HHHHH HHMOO) (11,0,0);
18:18 (SIII) (T) (EEEEH HHHHH HHHHH HHHMO) (13,0,0);
18:19 (TSII) (T) (EEEEH HHHHH HHHHH HHHHM) (11,0,0);
18:20 (STII) (S) (EEEEE EEEHH HHHHH HHHHH) (10,0,0);
18:21 (TIII) (S) (EEEEE EEEEE HHHHH HHHHH) (9,0,0);
18:22 (IIII) (S) (EEEEE EEEEE EHHHH HHHHH) (9,0,0);
18:23 (IIII) (S) (EEEEE EEEEE EEHHH HHHHH) (8,0,0);
18:24 (IIII) (S) (EEEEE EEEEE EEEHH HHHHH) (7,0,0);
18:25 (IIII) (S) (EEEEE EEEEE EEEEE HHHHH) (5,0,0);
18:26 (IIII) (I) (EEEEE EEEEE EEEEE HHHHH) (5,0,0);
18:27 (IIII) (I) (EEEEE EEEEE EEEEE HHHHH) (5,0,0);
18:28 (IIII) (I) (EEEEE EEEEE EEEEE HHHHH) (5,0,0);
18:29 (IIII) (S) (EEEEE EEEEE EEEEE EEHHH) (3,0,0);
18:30 (IIII) (I) (EEOEE EEEEE EEEEE EEHHH) (3,0,0);
18:31 (IIII) (T) (EEMEE EEEEE EEEEE EEHHH) (2,0,0);
18:32 (SIII) (S) (EOHEE EEEEE EEEEE EEEEH) (1,0,0);
18:33 (TIII) (T) (EMHEE EEEEE EEEEE EEEEH) (0,0,0);
18:34 (SIII) (S) (EHHOE EEEEE EEEEE EEEEE) (1,0,0);
18:35 (TIII) (T) (EHHME EEEEE EEEEE EEEEE) (1,0,0);
18:36 (SIII) (I) (EHHHE EEEEE EEEEE EEEEE) (2,0,0);
18:37 (TIII) (I) (OHHHE EEEEE EEEEE EEEEE) (2,0,0);
18:38 (IIII) (T) (MHHHE EOEEE EEEEE EEEEE) (3,0,0);
18:39 (SIII) (T) (HHHHE EMEEE EEEEE EEEEE) (3,0,0);
18:40 (TSII) (I) (HHHHE EHEEE EEEEE EEEEE) (3,0,0);
18:41 (ITII) (I) (HHHHE EHEEE EEEEE EEEEE) (4,0,0);
18:42 (IIII) (I) (HHHHE EHEEE EEEOE EEEEE) (5,0,0);
18:43 (IIII) (T) (HHHHE EHOEE EEEME EEEEE) (5,0,0);
18:44 (SIII) (I) (HHHHE EHOEE EEEHE EEEEE) (5,0,0);
18:45 (TIII) (T) (HHHHE EHMEE EEEHE OEEEE) (4,0,0);
18:46 (SIII) (T) (HHHHE EHHEE EEEHE MEEEE) (5,0,0);
18:47 (TSII) (S) (HEEHE EHHEE EOEHE HEEOE) (4,0,0);
18:48 (ITII) (I) (HEEHE EHHEE EOEHE HEEOE) (5,0,0);
18:49 (IIII) (T) (HEEHE OHHOE EMEHE HEEOE) (5,0,0);
18:50 (SIII) (T) (HEEHO MHHME EHEHE HEEOE) (5,0,0);
18:51 (TSSI) (S) (HEEEO HHHHE EHEHE HEEOE) (5,0,0);
18:52 (ITTI) (S) (EEEEO HHHHE EHEHE HEEOO) (5,0,0);
18:53 (IIII) (S) (EEEEO HEHHE EHEHE HEEOO) (6,0,0);
18:54 (IIII) (I) (EEEEO HEHHO EHEHE HEEOO) (6,0,0);
18:55 (IIII) (T) (EEEEM HEHHM EHEHE HEEOO) (6,0,0);
18:56 (SSII) (I) (EEEEH HEHHH EHEHE HEEOO) (6,0,0);
18:57 (TTII) (T) (EEEEH HEHHH EHEHE HEEMM) (5,0,0);
18:58 (SSII) (S) (EEEEH HEHHH EHOEE HEEHH) (7,0,0);
18:59 (TTII) (S) (EEEEH HEEHH EHOEE HEEHH) (6,0,0);
19:00 (IIII) (T) (EEEEH HEEHH EHMEE HEEHH) (7,0,0);
19:01 (SIII) (S) (EEEEH HEEHH EHHEE EEEHH) (7,0,0);
19:02 (TIII) (I) (EEEEH HEEHH OHHEE EEEHH) (7,0,0);
19:03 (IIII) (S) (EEEEH HEEHH OEHEE EEEHH) (7,0,0);
19:04 (IIII) (T) (EEEEH HEEHH MEHEE EEEHH) (5,0,0);
19:05 (SIII) (S) (EEEEH EEEEH HEHEE EEEHH) (5,0,0);
19:06 (TIII) (I) (EEEEH EEEEH HEHEO EEEHH) (5,0,0);
19:07 (IIII) (I) (EEEEH EEEEH HEHEO EEEHH) (6,0,0);
19:08 (IIII) (T) (EEEEH EEEEH HEHEM EEEHH) (6,0,0);
19:09 (SIII) (S) (EEEEE EEEEE HEHEH EOEHH) (4,0,0);
19:10 (TIII) (T) (EEEEE EEEEE HEHEH EMEHH) (4,0,0);
19:11 (SIII) (S) (EEEEE EEEEE HEHEH EHEEE) (3,0,0);
19:12 (TIII) (I) (EEEEE EEEEE HEHEH EHOEE) (3,0,0);
19:13 (IIII) (T) (EEEEE EEEEE HEHEH EHMEE) (4,0,0);
19:14 (SIII) (S) (EOEOE EEEEE HEEOH EHHEE) (3,0,0);
19:15 (TIII) (T) (EMEOE EEEEE HEEOH EHHEE) (3,0,0);
19:16 (SIII) (T) (EHOME EOEEE HEEOH EHHEE) (4,0,0);
19:17 (TSII) (T) (EHMHE EOEEE HEEOH EHHEE) (4,0,0);
19:18 (STII) (S) (EHHHE EOEEE EEEOH EHHEE) (4,0,0);
19:19 (TIII) (I) (EHHHE EOEEE EEEOH EHHEE) (5,0,0);
19:20 (IIII) (T) (EHHHE EMEEE EEEMH EHHEE) (6,0,0);
19:21 (SSII) (I) (OHHHE EHEEE EOEHH EHHEE) (6,0,0);
19:22 (TTII) (T) (MHHHE EHEEE EOEHH EHHEE) (5,0,0);
19:23 (SIII) (T) (HHHHE EHEEE EMEHH EHHOE) (7,0,0);
19:24 (TSII) (T) (HHHHE EHEEE EHEHA EHHME) (6,0,0);
19:25 (STII) (S) (HHHHE EHEEE EHEHE EEHHE) (7,0,0);
19:26 (TIII) (I) (HHHHE EHEEE EHEHE EEHHE) (8,0,0);
19:27 (IIII) (S) (HHHHE EHEEE EHEHE EEEHE) (8,0,0);
19:28 (IIII) (I) (HHHHE EHEEE EHEHE EEEHE) (8,0,0);
19:29 (IIII) (S) (HEHHE EHOEE EHEHE EEEHE) (7,0,0);
19:30 (IIII) (S) (HEHEE EHOEE EHEHE EEEHE) (6,0,0);
19:31 (IIII) (T) (HEHEE OHMEE EHEHE EEEHE) (5,0,0);
19:32 (SIII) (T) (HEHEE MHHEE EHEHE EEEHE) (5,0,0);
19:33 (TSII) (S) (HEEEE HHHEE EHEHE EEEHO) (5,0,0);
19:34 (ITII) (T) (HEEEE HHHEE EHEHE EEEHM) (4,0,0);
19:35 (SIII) (S) (HEEEE HEHEE EHEEE EEEHH) (5,0,0);
19:36 (TIII) (S) (EEEEE HEHEE EHEEE EEEHH) (4,0,0);
19:37 (IIII) (S) (EEEEE HEHEE OEEEE OEEHH) (4,0,0);
19:38 (IIII) (T) (EEEEE HEHEE MEEEE MEEHH) (3,0,0);
19:39 (SSII) (S) (EEEEE HEHEO HEEEE HOEEH) (3,0,0);
19:40 (TTII) (T) (EEEEE HEHEM HEEEE HMEEH) (3,0,0);
19:41 (SSII) (I) (EEEEE HEHEH HEEEE HHEEH) (5,0,0);
19:42 (TTII) (I) (EEEEO HEHEH HEEEE HHEEH) (5,0,0);
19:43 (IIII) (I) (EEEEO HEHOH HEEEE HHEEH) (7,0,0);
19:44 (IIII) (T) (EEEEM HEHOH HEEEE HHEEH) (7,0,0);
19:45 (SIII) (S) (EOEEH HEEOH HEOEE HHEEH) (6,0,0);
19:46 (TIII) (S) (EOEEH EEEOH HEOEE HHEEH) (5,0,0);
19:47 (IIII) (I) (EOEEH EEEOH HEOEE HHEEH) (6,0,0);
19:48 (IIII) (S) (EOEEH EEEOH HEOEE HHEEE) (5,0,0);
19:49 (IIII) (I) (EOEEH EEEOH HEOEE HHEEE) (5,0,0);
19:50 (IIII) (I) (EOEEH EEEOH HEOEO HHEEE) (5,0,0);
19:51 (IIII) (I) (EOEEH EEEOH HEOEO HHEEE) (5,0,0);
19:52 (IIII) (S) (EOEEH EEEOH EEOOO EHEEE) (3,0,0);
19:53 (IIII) (I) (EOEEH EEEOH EEOOO EHEOE) (3,0,0);
19:54 (IIII) (S) (EOOEH EEEOE EEOOO EEEOE) (1,0,0);
19:55 (IIII) (I) (EOOOH EEEOE EEOOO EEEOE) (1,0,0);
19:56 (IIII) (I) (EOOOH EEEOE EEOOO EEEOE) (1,0,0);
19:57 (IIII) (I) (EOOOH EEEOE EEOOO EEEOE) (1,0,0);
19:58 (IIII) (S) (EOOOE EOEOE EEOOO EEEOE) (0,0,0);
19:59 (IIII) (I) (EOOOE EOEOE EEOOO EEEOE) (0,0,0);
20:00 (IIII) (I) (EOOOE EOEOE EEOOO EEEOE) (0,0,0);
\end{lstlisting}
\bibliographystyle{plain}
\bibliography{bibitex}

\appendix
\section*{Appendix: Code Package}
\subsection{Directories}
./code:
prob1  prob2  prob3

./code/prob1:
prob1\_i\_ijk\_static.c  prob1\_j\_ijk\_static.c  prob1\_j\_jik\_static.c  prob1\_k\_ijk\_static.c  prob1\_k\_kji\_static.c

./code/prob2:
prob2\_dynamic.c  prob2\_guided.c  prob2\_static.c

./code/prob3:
pizza\_new.cpp

\subsection{About Problem 1}
\texttt{prob1\_\$v\_\$seq\_\$schedule.c}: \$v is denoted for the identifier of loop which is paralleled; \$seq is the order of loop, \$schedule contains static, dynamic, guided. 

\end{spacing}
\end{CJK}
\end{document}
